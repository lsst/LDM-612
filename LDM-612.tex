% vim: tw=0:wrap:linebreak
\documentclass[DM,toc,lsstdraft]{lsstdoc}

\usepackage{comment}
\usepackage{datetime}
\usepackage{microtype}

\newcommand{\microarcsec}{$\mu$as\xspace}
\interfootnotelinepenalty=10000


\setcounter{secnumdepth}{3}

%%%%%%%%%%%%%%%%%%%%%
% Introduce mechanism to turn on and off various annotations (\XXX command,
% \begin{notes}, and anything else bracketed by \ifannotated ... \fi)
%
\newif\ifannotated
\annotatedtrue
\annotatedfalse	% uncomment this to hide all annotations (comments, notes, etc)

\ifannotated
	% leave things as-is
\else
	% hide all \XXX commands
	\renewcommand{\XXX}{}

	% hide all \begin{note}...\end{note} text
	\renewenvironment{note}[1][Note]
	{}
	{}
\fi
%%%%%%%%%%%%%%%%%%%%%

\title{Plans and Policies for LSST Alert Distribution}
\author{
     Eric C. Bellm, Melissa L.~Graham, Beth Willman, Zeljko Ivezi\'{c}, William O'Mullane, and Maria T.~Patterson
     \emph{for the LSST Project}
}

\setDocRef{LDM-612}
\setDocCurator{E.C.~Bellm}
\date{\today}

\setDocAbstract{%
A major product of the nightly processing of LSST images is a world-public stream of alerts from transient, variable, and moving sources.
Science users will access these alerts through community brokers or through a simple filtering service provided by LSST.
This document provides a guide to the \oldtext{planned} \newtext{plans and policies for the } alert distribution system to aid science users, broker developers, funding agencies, and LSST Project personnel.
It describes the components of the alert distribution system and the data rights required to access specific scientific products.
It provides guidelines for organizations developing community brokers \oldtext{.
And it} \newtext{and} describes the planned capabilities of the LSST simple alert filtering service and Science Platform to aid science users in planning for LSST science.
}

%
%   Revision history
%
% OLDEST FIRST: VERSION, DATE, DESCRIPTION, OWNER NAME
\setDocChangeRecord{%
\addtohist{1}{2018-06-23}{Initial version}{Eric Bellm}
}


\begin{document}

\maketitle

\section{Introduction}\label{sec:introduction}

LSST will discover large numbers of astrophysical objects that move, change, or appear.
Because of their intrinsic temporal variability, full scientific exploitation of many classes of these events requires rapid discovery and additional follow-up observations.
Accordingly, LSST's real-time Prompt Processing pipelines will identify such detections using difference imaging and report them using world-public \textit{alerts} issued within 60 seconds of the shutter closure.

These alert packets will contain not only the information about the most recent detection but also a historical lightcurve, cutout images, timeseries features, and other contextual information.
An LSST science user will be able to use the contents of a single alert packet and make an rapid but informed decision about whether the event is relevant to their scientific goals.
LSST expects to produce about ten million of these alerts nightly.
The resulting \textit{alert stream} will be large---more than 1\,TB nightly---but will contain all classes of astrophysical events, from the youngest supernovae to the most distant RR Lyrae to slow-moving Trans-Neptunian Objects.
The task for the scientist is to identify the small subset of events of interest in the larger alert stream.

To do so, astronomers will rely on third party \textit{community brokers}, software systems that receive the full LSST alert stream and add additional information to it.
Community brokers may crossmatch the LSST stream to multiwavelength catalogs, join LSST alerts with those from other surveys, provide machine-learned classifications of events, and/or offer user filters to winnow the stream.
LSST will itself offer a simple filtering service, the ``mini-broker,'' that will apply user-supplied filters to the alert stream.
Individual scientists will receive alerts through one or more of these services.

The large volume of the alert stream and the finite bandwidth from the LSST Data Facility necessitates a proposal process to select community brokers to receive the full stream.
This document outlines the process, criteria, and timeline by which LSST will choose community brokers.
To provide context, we also summarize the major features of the LSST Alert Production systems as well as relevant data rights concerns.

Throughout the document, we provide references to the formal LSST requirements and design documents as an aid to the interested reader.

\textit{
Goal: In narrative language, outline LSST plans and policies for alert distribution for a broad audience.

Audience: LSST science users, community broker developers, funding agencies, LSST Project personnel

Venue:  LDM document (LDM-612), to be posted on the arxiv after CCB acceptance. Potentially a living document

Timeline: draft to SAC by Spring 2018; CCB acceptance by AHM 2018?

{\it Common Questions:} MLG has ported all of her ``commonly asked questions from the community" from the Google Doc into their relevant section of this document.

Note that the SAC had several recommendations for what this broker doc should contain.
\url{https://project.lsst.org/groups/sac/sites/lsst.org.groups.sac/files/2017Aug14_minutes.txt}

The five main requests made by the SAC in August 2017 are:
\begin{itemize}
\item The plans for giving guidelines for and selecting event brokers
  from the scientific community need to reflect the fact there is
  likely to be a continuum of needs: from community brokers designed
  to serve the world, to specialized filters aiming to do specific
  scientific projects.  We recommend that the call for brokers reflect
  this continuum. Guidelines need to be set to describe how many of these specialized
brokers can be accommodated, and what the rules will be.
\item We recommend that the project start a conversation with the
  Transients and Variable Stars science collaboration about use cases
  for brokers, to understand the extent to which the LSST simple
  broker, and a few general community brokers, will or will not
  satisfy the needs of the community. 
\item The SAC remains unclear on whether the outputs of brokers can, or should, be world-public. The SAC remains unclear on exactly what this would include. Will the community brokers have the ability to query the database for additional information (which may be proprietary to the US, Chile, and International Contributors)? If so, do they have the right to share the results with the world? This needs to be clarified.
\item There will also be a desire to run queries on archival data (not just on the real-time alert stream), for statistical studies of various categories of transients.  It remains unclear whether this will be possible.
\item The SAC looks forward to reviewing the document which will describe the requirements and review process for brokers. It should include the proposal process, timescale, and
review process for groups to request access to the raw data stream.
Exactly what will the data stream these brokers will receive include?
What level of support, if any, will the brokers receive from the
Project during construction, commissioning, and operations?  Will the
community or specialized brokers be expected to archive their results
for posterity, and if so, how should/could this be integrated with the
LSST database and science user interface?  Will the answers to these
questions depend on the scale of the broker being proposed?
\end{itemize}

Continue to check on the relevant community concerns that are likely to be voiced here:
Community discussion on LSST Alert Broker (categories Science and TVS SC)
\url{https://community.lsst.org/t/lsst-alert-brokers/2286}

Related documents: existing requirements documents (SRD, DMSR, DPDD, etc.) used as input. Future documents might also include:
\begin{itemize}
\item{New Alert Distribution Requirements document (possibly part of L1 Requirements Doc; in formal requirements language)}
\item{New Alert Distribution Test Plan (possibly part of L1 Test Plan)}
\item{New Alert Distribution Design Document (Analogous to LDM-151 for Science Pipelines)}
\end{itemize}

Paragraph of overview for sections?

Conventions used in this document, such as the use of {\tt tt text formatting}, using {\tt Source} for a detection in a single image and {\tt Object} for a given location on the sky, etc. Direct readers to Glossary in Section \ref{sec:glossary}.
}


\section{Components and Capabilities of the LSST Alert Distribution System}\label{sec:components}

In this section we provide a high-level overview of the LSST Alert Production process to provide background for science users of LSST alerts.
Key numbers for Alert Production are summarized in \citeds{DMTN-102}.

\subsection{Facilities}

\subsubsection{Data Acquisition, Transfer, and Processing}

LSST will survey the sky repeatedly using standard or \newtext{``}alternative standard\newtext{''} visits (currently baselined as $2\times15$\,second exposures in a single band or a single 30\,second exposure, respectively).
Immediately after the camera shutter closes, the images will be transferred via fiber networks to the LSST Data Facility (LDF; \S \ref{sec:LDF}) at the National Center for Supercomputing Applications (NCSA) in Illinois.
All subsequent data processing leading to alerts occurs in the LDF.

\subsubsection{The LSST Data Facility and Data Access Centers} \label{sec:LDF}

\citeds{LDM-230} describes the concept of operations for the LSST Data Facility.
The LDF hosts a variety of computational services, including the Prompt Processing services that run the Alert Production pipelines (\S \ref{sec:AP}) and the Batch Production services that run the annual Data Release Processing (\S \ref{sec:drp}).
All Prompt Processing and Alert Production runs at the LDF;
Data Release Processing is split between the LDF and the CC-IN2P3 in France.
The LDF also hosts a Data Access Center (DAC) that provides access to proprietary data products (\S \ref{sec:data_rights}) through the LSST Science Platform (\S \ref{sec:LSP}).
The LDF also hosts the alert stream feeds to community brokers (\S \ref{sec:alert_distribution}) \newtext{and the LSST Alert Filtering Service (\S \ref{sec:filtering_service})}.

An additional DAC will be located in Chile.
Its capabilities are described in \citeds{LDM-572}.
It will host a copy of the raw data and processed data products and run a version of the Science Platform.

\subsection{Pipelines and Services}

\subsubsection{Alert Production}\label{sec:AP}

LSST Data Management (DM) Alert Production (AP) processes data during the night and generates alerts \newtext{in close to real-time}.
It is described in \citeds{LSE-163}, with detailed descriptions of its constituent pipelines and algorithmic components provided in \citeds{LDM-151}.
The basic modules include single-frame image processing (e.g., instrument signature removal, photometric and astrometric calibration), difference image analysis (DIA; the subtraction of a coadded template image constructed from the most recent Data Release from the new visit image), and source detection, association, and measurement \citedsp{LSE-163}.
AP will process all standard and alternative standard visits, whether obtained as part of the main wide-fast-deep survey or a Special Program \citedsp{DMTN-065} and generate alerts within the required 60 seconds after the shutter closes (DMS-REQ-0004, \citeds{LSE-61})\footnote{
AP may also be able to process non-standard visits with longer or shorter exposure times provided that, e.g., the visit image can be successfully PSF-matched and differenced with a template image.
Alert generation in crowded fields may produce more than the maximum of 10,000 alerts per visit required to be supported by DM; alert generation in this circumstance is still under study.
}.
Due to the need for Data Release Production-derived templates, Alert Production cannot run at full scale and full fidelity during commissioning nor the first year of operations.  LSST DM is currently investigating options for Alert Production in year one \newtext{\citedsp{DMTN-107}}.
%LSE-459

All sources in a difference image that have a signal-to-noise ratio ${\rm SNR}>5$ in positive or negative flux are considered ``detected" \texttt{DIASources}, are incorporated into the {\tt DIASource} catalog, and will cause an alert to be issued\footnote{
There may be exceptions to this rule, such as the following two examples.
(1) Sources with ${\rm SNR} > 5$ that have a {\it ``high probability of being instrumental non-astrophysical artifacts"} \citedsp{LSE-163}, potentially as determined by
a to-be-developed spuriousness or real/bogus classifier \citedsp{LDM-151}, may not produce alerts.
The limit on false positives is set by the requirement that the alert stream be 90\% complete at 95\% purity for \texttt{DIASources} of SNR $= 6$ \citeds{LSE-61}.
(2) Sources with ${\rm SNR} < 5$ that meet other to-be-determined criteria, such as the likelihood of being a potentially hazardous asteroid, could produce alerts.}.
Additionally some ${\rm SNR} < 5$ sources will be kept for diagnostic purposes, but will not lead to alerts.
Source association and measurement occurs prior to alert generation: every {\tt DIASource} will have one unique match to a {\tt DIAObject} (stationary object) or known Solar System Object {\tt SSObject} (moving object; see Section \ref{sec:AGP_MOPS}).
The stationary source association algorithms will be probabilistic and incorporate motion models for parallax and proper motion \citedsp{LDM-151}.
If no association is possible a new {\tt DIAObject} will be created.
Source measurements including centroid, fluxes, shapes, and other characterization parameters (Table 1 of \citeds{LSE-163}) are made, and the time-evolving parameters of the {\tt DIAObject} such as the parallax, mean flux, and periodic/non-periodic light curve features, will be updated to include the new associated {\tt DIASource}.
At this point, the {\tt DIASource} will be used to create an alert packet (\S \ref{sec:packets}), regardless of whether the \texttt{DIASource} is associated with a new \texttt{DIAObject}, a previously-existing \texttt{DIAObject}, or an \texttt{SSObject}.

\subsubsection{Moving Objects Processing System}\label{sec:AGP_MOPS}

After the completion of a night of observing,
the LSST Moving Objects Processing System (MOPS; \citeds{LDM-156}) will attempt to identify \textit{new} \texttt{SSObjects} using the updated Prompt Products Database (\S \ref{sec:ppdb}).
MOPS algorithms form tracklets from pairs of single-apparition \texttt{DIASources} taken during one night, generate tracks between nights, and fit orbits to identify and characterize moving objects.
Newly identified moving objects are added to the {\tt SSObject} catalog.
MOPS will \oldtext{interface with}\newtext{report candidate moving objects to} the Minor Planets Center (MPC)\newtext{, where they will be publicly available;} will ingest all previously known or externally identified moving objects into the {\tt SSObjects} catalog\oldtext{,}\newtext{;} and will use MPC astrometry in the orbital parameter fits.
Users interested in newly-discovered asteroids will need to query the \texttt{SSObject} catalog or the MPC to identify them.
The first time a new \texttt{SSObject} is discovered, MOPS will associate past \texttt{DIAsources} with the new \texttt{SSObject} as far back as is practical from the accuracy of the orbit.
However, revised alerts will not be issued.
Future LSST detections of the new \texttt{SSObject} will produce alerts attributed to the \texttt{SSObject}.


\subsubsection{Alert Distribution} \label{sec:alert_distribution}

Alert packets (\S \ref{sec:packets}) will be queued for distribution to community brokers (\S \ref{sec:community_brokers}) and the LSST alert filtering service (\S \ref{sec:filtering_service}).
LSST is prototyping  a bulk transport system built on the open-source distributed queue system Apache Kafka\footnote{\url{http://kafka.apache.org/}}, with Apache Avro\footnote{\url{http://avro.apache.org/}} used as a binary serialization format \newtext{\citedsp{DMTN-093}}.
Initial tests indicate that this system performs effectively at the required scale \citedsp{DMTN-028}.


An allocation of 10\,Gbps is baselined for alert stream transfer from the LDF, with an estimated packet size of 82\,KB and up to 10,000 alerts per visit.
Due to the finite bandwidth out of the LDF, only a limited number of selected community brokers can receive the full alert stream.
It will not be possible for individual science users to subscribe directly to the full LSST alert stream.
Rather, science users may access LSST alerts through a community broker, or through the alert filtering service (\S \ref{sec:filtering_service}) if they have data access rights (\S \ref{sec:data_rights}).
\oldtext{For illustration, based on these numbers up to 7 brokers could receive the full stream if 5\,seconds is budgeted for outbound data transfer.}
\newtext{A minimum of five full streams is required to be supported \citedsp{LSE-61}.}
The eventual number of selected community brokers (\S \ref{sec:numbrokers}) is still to be determined, as it depends on several quantities including available network capacity, packet size, number of alerts per visit, and the rate of alert distribution.
\oldtext{The broker selection process described in \S \ref{sec:community_brokers} is most appropriate for the current estimate.
If only one or two full streams could be supported more explicit requirements on the selected brokers would be needed; if tens of full streams could be supported a simpler selection process could be used.}


\subsubsection{The LSST Alert Filtering Service}\label{sec:filtering_service}

The LSST alert filtering service provides a means of real-time access to a subset of the alert stream.
Using the Science Platform (\S \ref{sec:LSP}),
users can upload simple filters (filters which only use information within the alert packet) and receive on their own computer a real-time stream of alerts which pass their filter.
This capability is expected to be useful for science cases requiring real-time notification that may be too specific to be well-served by community brokers.
The alert filtering service is required to support a minimum of 100 simultaneous users; limits on resource usage and bandwidth (support for transmission of up to 20 full-sized alerts per visit per filter is required) will be imposed to maintain performance\footnote{
The requirement on the number of simultaneously connected users and number of passed alerts is largely driven by outbound bandwidth limitations from the DAC at NCSA.
We are investigating approaches that would support larger numbers of active filters.
Users with data rights that do not require real-time notification can also programatically query the Prompt Products Database to identify \texttt{DIAObjects} or \texttt{SSObjects} of interest without using the LSST alert filtering service.}.
If necessary, a TAC-like process may be used to allocate filtering service resources \citedsp{LSE-163}.
The LSST alert filtering service will support a version of the \texttt{VOEvent} standard\footnote{\url{http://www.ivoa.net/documents/VOEvent/}} that is current at the time of LSST operations.

\subsubsection{Forced Photometry}\label{sec:AGP_force}

Measurements of \texttt{DIAObjects} that are below the ${\rm SNR} > 5$ threshold are available through ``forced photometry'': flux is measured at a previously-known position.
Forced photometry is performed on difference images in two ways; neither contributes to alerts, but are available to users through the Science Platform.
First, during the real-time difference image analysis, forced photometry is performed on every difference image at the locations of all previously-known {\tt DIAObjects} detected in a past interval to be determined.
The resulting measurements are stored in the Prompt Products Database (\S \ref{sec:ppdb}) and are queryable within 24 hours.
Second, at the end of the night, ``precovery'' forced photometry will be performed for all \textit{new} \texttt{DIAObjects} on every difference image from the past 30 days.
These results are also stored in the Prompt Products database and will be available within 24 hours.

Additionally, a service shall be provided to users to obtain full-survey precovery for a limited number of user-specified {\tt DIAObjects}, also within 24 hours (DMS-REQ-0341, \citeds{LSE-61}).
This service will be made available through the Science Platform.

\subsubsection{The LSST Science Platform} \label{sec:LSP}

The LSST Science Platform (LSP) provides access to the proprietary data products (e.g., \S \ref{sec:products}--\ref{sec:alertdb}) held in the DACs.
The Science Platform is described in full in \citeds{LSE-319} and \citeds{LDM-554}.
It provides three means of accessing data: a web-based ``Portal'' for visual examination and querying of the LSST images and catalogs;
an interactive ``Notebook'' environment with associated computing allocations for running code close to the data;
and an Application Programming Interface (API) for programmatic access to LSST data using Virtual Observatory standards.

\subsection{Data Products}
\subsubsection{Alert Packets}\label{sec:packets}

The contents of an {\tt Alert} packet are fully described in Section 3.5.1 of \citeds{LSE-163}. We reproduce the list of included data here:
\renewcommand{\labelenumi}{\Roman{enumi}.}
\begin{enumerate}
\item an ID uniquely identifying the {\tt Alert}
\item the prompt products database ID
\item the database record of the {\tt DIASource} that triggered the {\tt Alert}, as well as the \texttt{filterName} and \texttt{programId} of the corresponding \texttt{Visit}
\item the entire associated {\tt DIAObject} or {\tt SSObject} record from the Prompt Products Database, which include a variety of variability metrics computed on the
updated \texttt{DIASource} lightcurve.
\item the previous 12 months of associated {\tt DIASource} records from the Prompt Products Database
\item matching {\tt Object} IDs from the latest Data Release, if they exist, and 12 months of their {\tt DIASource} records
\item postage stamps of the difference image and template at the {\tt DIASource} location
\end{enumerate}

The full list of parameters that are measured and included in the {\tt DIASource} and {\tt DIAObject} records are provided in Tables 1 and 2 of \citeds{LSE-163}, respectively.
%The average size of an {\tt Alert} packet will be {\bf XXX?} kilobytes and {\bf XXX?} \% of this is the postage stamp.
The only trigger for an {\tt Alert} is the detection of a source in a difference image with ${\rm SNR} > 5$.
Thus, objects that are saturated in the visit or template image, or objects that are the same brightness in both the visit and template images will not generate an {\tt Alert}.
Imperfect subtractions may create artifacts with ${\rm SNR} > 5$;
these will generally lead to an \texttt{Alert}, subject to overall capacity limits.
A machine-learned spuriousness score will be provided for each \texttt{DIASource} triggering an \texttt{Alert}, so users may filter the stream for greater completeness or purity depending on their scientific needs.
The spuriousness classifier is required to achieve
90\% completeness and 95\% purity at 6$\sigma$ \citedsp{LSE-30} at a threshold spuriousness value.


\subsubsection{Processed Images} \label{sec:products}

Processed visit images (PVIs) and difference images produced by the prompt processing pipeline will be available in the Science Platform to users with data rights within 24 hours.
They are held on disk for 30 days to enable the precovery forced photometry service (\S \ref{sec:AGP_force}).
PVIs for older visits can be regenerated on-demand.

\subsubsection{Prompt Products Database} \label{sec:ppdb}

The \texttt{DIAObject}, \texttt{DIASource}, and \texttt{SSObject} catalogs created by the Alert Generation pipelines are held in a Prompt Products Database along with the outputs of the Forced Photometry.
This database is queryable through any of the aspects of the Science Platform.


\subsubsection{Data Release Products} \label{sec:drp}

Annual\footnote{There will be an additional initial data release containing data from the first six months of operations \newtext{\citedsp{LSO-011}}.}  releases of all the LSST data will include processed and stacked images, catalogs of {\tt Sources}, {\tt ForcedSources}, and {\tt Objects} from measurements on the stacked and individual images, as well as calibration information.
It will also include a  reprocessing of all images with the latest pipelines and full-survey versions of {\tt DIASource} and {\tt DIAObject} catalogs---i.e., the variability characterization parameters are calculated from the entire survey to date.
All of these products are queryable through the Science Platform.

\subsubsection{Alert Database} \label{sec:alertdb}

All alerts will be stored in their full original form in an archive hosted in the DACs.
This archive is expected to be of most use for testing broker filters or for studies that need to emulate real-time identification of transients from the alert packets.
Users who simply wish to work with large samples of lightcurves will likely prefer to query the \texttt{DIAObject} and \texttt{DIASource} or \texttt{Object} and \texttt{ForceSource} tables directly.


\section{Data Rights to Alert Stream Components}\label{sec:data_rights}

\textbf{LPM-216 is not yet change-controlled and so the following should be regarded as provisional until it is. Broken references are to LPM-216.} The proposed LSST data rights and data access policies (DAPOLs) are formally described in \citeds{LPM-216}. Here we summarize the aspects most relevant to the alert stream. The interested reader is advised to consult \citeds{LPM-216} for details.

LSST {\tt Alert} packets (\S \ref{sec:packets}) are a world-public data product.
We use ``public'' here in the sense of DAPOL-020:  \texttt{Alert} packets can be freely shared with anyone, by anyone, anywhere, worldwide.
However, LSST is not committed to serving alert packets directly to individuals without data rights;
thus, the term ``public'' means ``shareable'' and should not be misinterpreted as ``freely available.''

Instead, the LSST {\tt Alert} Stream will be delivered to a to-be-determined set of community brokers (Section \ref{sec:community_brokers}).
An institution is not required to hold data rights in order to host a broker.
Brokers may share (or not share) the contents of {\tt Alert} packets with whomever they choose, 
although the broker selection process (\S \ref{sec:community_brokers}) will prioritize selection of brokers that will provide world-public access.
Brokers will not be prohibited from storing {\tt Alerts} and making them available at later times;
the {\tt Alert} packets themselves are world-public indefinitely, and thus aggregations of public products are also permitted.

Essentially all other LSST data products and services are proprietary, including raw and processed images, the resulting catalogs and databases, and access to the LSST DAC, Science Platform, and mini-broker.

Brokers hosted by institutions with LSST data rights and data access (DAPOL-040) may access proprietary LSST data products, such as through API interfaces exposed by the LSST Science Platform (\S \ref{sec:LSP}).
The broker would then be responsible for ensuring access to the proprietary data products is restricted to data rights holders (DAPOL-020).
Any \textit{derived} data products that are generated from proprietary data are also proprietary until published, after which they are considered public. 
DAPOL-620 defines derived data products as data that cannot be used to recreate any proprietary LSST data products.

For example, a community broker could use DAC APIs to obtain the proprietary full-history forced-photometry light curve for a \texttt{DIAObject} identified by a public {\tt Alert}, 
and then use this information to construct a classification probability that represents the likelihood the object is a tidal disruption event.
The classification probability would, under the currently proposed data rights policies, be classified as a derived data product that is proprietary until published. The light curve itself remains proprietary.

%A common question about broker DAC privileges: Will there be a way for brokers to make products available through the DAC, so that authorization (in the case of brokers using proprietary data products) can be handled by LSST's protocols?
%Another question about broker DAC privileges: Brokers that are accessing the DAC through a web API during the night might use a large amount of the DAC's computational resources -- has this been sized yet and how would decisions to throttle usage be made, if it became necessary to do so?

Full LSST Users (DAPOL-080) will have access to the mini-broker in the LSST DAC for filtering the alert stream.
Since the alerts themselves are world public, a Full LSST User who uses the mini-broker to process the alert stream can export filtered alert packets (or any subset of their contents) and share them freely.
Full LSST Users may use other DAC services to query and process proprietary data products. 
For example, a user might identify a \texttt{DIAObject} of interest from a mini-broker filter, query the Prompt Products Database to retrieve the precovery forced photometry, and then run user-generated processing on the Processed Visit Images.
These proprietary products must not be distributed to individuals without data rights.
As for community brokers, if a user generates derived data products (such as a classification probability derived from a fit to the forced photometry lightcurve), these also remain proprietary until published. However, exceptions for the sharing of derived data products in advance of publication are described by DAPOL-720: in short, instances in which sharing uniquely enables a publication or preserves the value of data rights.

As an example, we consider a Full LSST user who is searching for supernovae using the mini-broker.
The LSST user wants to share a target with a collaborator who does not have data rights so that the collaborator may obtain a spectrum. This is justified under DAPOL-720 if the spectrum will uniquely enable a publication (e.g., supernova data do not typically yield a publishable unit without spectroscopic classification). The LSST user may send their collaborator the sky position and time of the first LSST detection (public from the \texttt{Alert} packet), the alert packet cutout images (public from the \texttt{Alert} packet), and their assessment that the target is a likely supernova Ia (i.e., the derived data product). They may not share a larger cutout from the Processed Visit Image to use as a finder chart nor the forced photometry light-curve they used to derive the supernova classification (proprietary data products). 


\section{Guidelines for Community Brokers}\label{sec:community_brokers}

\subsection{The Role of Community Brokers}

Community brokers will play a vital role in enabling time-domain science with LSST.
By receiving the LSST alert stream, adding value, and redistributing it to the scientific community, brokers facilitate full exploitation of the scientific value of the real-time LSST alerts.

We envision that community brokers, individually or jointly, may provide a variety of functions.  These may include, but are not limited to:

\begin{itemize}
	\item redistribute alert packets
	\item filter alerts
	\item cross-correlate LSST alerts with other static catalogs or alert streams
	\item classify events scientifically
	\item provide user interfaces to the data
	\item coordinate scientific activity 
	\item trigger followup observing
	\item for users with appropriate data rights, facilitate followup queries and/or user-generated processing within the LSST Data Access Center
	\item manage annotation \& citation 
	\item collect classification and other information gathered by the scientific community
\end{itemize}

Some these functions may be performed by ``Target and Observation Managers'' that may either be integrated into community brokers directly or connect to annotated and/or filtered streams delivered by a broker.
For simplicity, we will here refer to any system that connects directly to the LSST alert stream as a broker.

\subsection{Requirements for Community Brokers}

Because of the large data volume of the alert stream (several TB per night), finite bandwidth from the LSST datacenter prevents sending a full copy of the alert stream to all interested parties.
Additionally, we anticipate that community brokers will require some level of support from the LSST operations team.
Accordingly, LSST will conduct an open proposal process in order to select brokers with sufficient technical capability to enable valuable and reliable scientific returns.

We believe the scientific community will be best served by a rich broker ecosystem offering diverse capabilities. 
Any institution worldwide may submit a proposal.
Demonstration of appropriate technical and personnel resources to support the proposed goals will be the minimum requirement to be considered.

Recognizing that different science goals may benefit from different technical approaches, and that proposing institutions may have a range of strengths and experience, the proposal call will not place any explicit functional requirements on proposed brokers.  
Thus a proposed broker is not \textit{required} to consume the whole stream, to redistribute the full stream, or to make its products world-public, for example.
However, given the small and finite number of brokers to be selected (\S \ref{sec:numbrokers}) and the evaluation criteria (\S \ref{sec:evaluation}), proposals offering to do so will be more likely to be selected.
The selection committee will evaluate the proposals holistically, weighing the unique capabilities proposed in order to maximize the scientific output of LSST.

While we expect the evaluation process to favor proposals that receive the full alert stream, brokers may propose to receive only a filtered subset of events (or a subset of the alert packet contents).  
We anticipate that this facility would be provided by the same technology underlying the LSST mini-broker, potentially with less restrictive requirements than are placed on user-provided filters. 

\subsection{The Broker Selection Panel}

Broker proposals will be evaluated by a panel designated by the LSST Science Advisory Committee.  Appropriate care will be taken to avoid conflicts of interest.
The panel will include ex officio representatives from the LSST Project Office and Data Management to provide policy and technical guidance.

\subsection{Required Technical Resources} \label{sec:resources}

The panel will evaluate whether the proposal is technically feasible and appropriately supported for the proposed goals.  
At minimum, this requires:

\begin{itemize}
	\item Large inbound and outbound network bandwidth (the full alert stream is a few TB/night)
	\item Petabytes of disk capacity
	\item Databases handling of billions of sources
	\item Compute resources to handle sophisticated classification and filtering tasks in real time at scale
	\item Appropriate personnel to develop and maintain the service
	\item Institutional support to ensure the longevity and stability of the service.
\end{itemize}

Depending on the relative timing of the proposal evaluation and the progress of LSST commissioning, the panel may request proof-of-concept execution on LSST commissioning datasets.

\subsection{Evaluation of Community Broker Proposals} \label{sec:evaluation}

The broker selection panel will evaluate proposals with a primary goal of maximizing the scientific utilization of LSST---recognizing that many of LSST's scientific returns are likely to be unanticipated.

The panel will evaluate broker proposals that demonstrate appropriate technical resources (\S \ref{sec:resources}) using the following evaluation criteria:

\subsubsection{Scientific Value}

The panel will consider whether a proposed broker has the potential to add scientific value that serves a large community, enables high-profile science, or provides unique capabilities.
In particular, the panel will evaluate the proposed contributions to LSST's four science pillars\footnote{Probing Dark Matter and Dark Energy, Taking an Inventory of the Solar System, Exploring the Transient Optical Sky, and Mapping the Milky Way}.
The panel will prioritize science cases that require or take advantage of the unique aspects of the LSST alert stream, namely its real-time and world-public nature.

\subsubsection{Availability to the World Community}

Because community brokers are the means by which scientists without LSST data rights may access the public contents of the alert packets, the panel will prefer proposals that make their products and services world-public.

\subsubsection{Scientific Validity}

The panel will look for evidence that scientific products, such as photometric classifications, produced by a proposed broker are accurate.  
Brokers may demonstrate performance in production on precursor surveys or by data challenge.
The proposal call will provide archived alert streams from precursor surveys that may be used for testing and evaluation.

\subsubsection{Integration with the Time-Domain Ecosystem}

For science cases involving real-time followup, the panel will consider the extent to which proposed brokers facilitate integration with followup resources, other surveys and alert streams, other brokers and services, existing communities of observers, and archives.  	


\subsubsection{Community Adoption}

The panel will consider whether a proposed broker system has already demonstrated scientific value and community adoption on precursor streams.  
This evaluation will also assess the 
range of scientific, geographic, and institutional communities that have utilized the service as well as the publications it enabled.


\subsubsection{Complementarity}

To maximize scientific returns, the panel will make a final selection of an ensemble of complimentary brokers rather than attempting to strictly rank disparate proposals.   Accordingly the final selection may include lower-ranked proposals if they provide capabilities not present in the top-ranked brokers or serve other scientific communities.

\subsubsection{Existing Agreements}

The panel will consider the applicability of any existing agreements, if relevant.

\subsection{Determining the Number of Community Brokers} \label{sec:numbrokers}

Initial sizing estimates indicated that only four full streams could be provided from the LSST Data Facility due to network bandwidth limitations.  
Current capacity projections indicate a larger number may be possible, but with some uncertainty.
The available capacity will depend on:

\begin{itemize}
	\item actual network capacity and usage patterns
	\item details of the alert format
	\item the achieved purity of the alert stream
\end{itemize}

The proposal call will provide updated estimates based on pipeline performance on precursor surveys.
Final capacity determination will not be possible until commissioning data is available from ComCam.
The broker selection panel may elect to make guaranteed and provisional selections based on these estimates.

\subsection{Resources for Proposers}

To aid in broker development and testing, at the time of the broker proposal call LSST will make available canned alert streams for precursor and/or simulated datasets generated by the LSST Alert Production pipelines and distributed with the expected formats and protocols.

\subsection{Resources for Selected Brokers}

Selected community brokers will receive basic support from the LSST Operations Team to ensure that the agreed to (portion of) the alert stream is pushed out with the expected form, content, reliability, and timing.
Additionally, status information will be available during operations to diagnose processing and data transport problems as they arise.

\subsection{Timeline}

\begin{enumerate}
  \setcounter{enumi}{2017}
	\item \begin{itemize}
		\item Plans and Policies for LSST Alert Distribution document issued
	\end{itemize}
	\item \begin{itemize}
		\item Document bulk transport format and interface
		\item Set up test alert stream with sample precursor data
		\item Issue call for proposals for community brokers
		\end{itemize}
	\item \begin{itemize}
		\item Broker proposals due
		\item Produce sample commissioning alerts with ComCam \textit{with substantial latency}
		\end{itemize}
	\item \begin{itemize}
			\item Finalize number of full streams
			\item Selection of community brokers for early operations
			\item Produce sample commissioning alerts with LSSTCam \textit{with substantial latency}
			\item Integrate selected brokers with the LSST Alert System
		\end{itemize}

	\item \begin{itemize}
			\item Begin full LSST operations
	\end{itemize}

\end{enumerate}

It is reasonable to expect periodic review of broker performance and usage during LSST Operations.


\clearpage

\section{Acronyms}
\addtocounter{table}{-1}
\begin{longtable}{|l|p{0.8\textwidth}|}\hline
\textbf{Acronym} & \textbf{Description}  \\\hline

AP & Alerts Production \\\hline
API & Application Programming Interface \\\hline
C & Specific programming language (also called ANSI-C) \\\hline
DAC & Data Access Center \\\hline
DAPOL & Data Access Policy (Tag for policy statement) \\\hline
DIA & Difference Image Analysis \\\hline
DM & Data Management \\\hline
DMS & Data Management Sub-system \\\hline
DMTN & DM Technical Note \\\hline
DRP & Data Release Production \\\hline
ID & Identifier (Identification) \\\hline
KB & KiloByte \\\hline
LDF & LSST Data Facility \\\hline
LDM & LSST Data Management (handle for controlled documents) \\\hline
LOI & Letter of Intent \\\hline
LPM & LSST Project Management (Document Handle) \\\hline
LSE & LSST Systems Engineering (Document Handle) \\\hline
LSP & LSST Science Platform \\\hline
LSST & Large Synoptic Survey Telescope \\\hline
MOPS & Moving Object Pipeline System \\\hline
MPC & Minor Planet Centre \\\hline
NCSA & National Center for Supercomputing Applications \\\hline
PSF & Point Spread Function \\\hline
RR & Rate Reduction \\\hline
SNR & Signal-to-Noise Ratio (also denoted SN and S/N) \\\hline
TAC & Time Allocation Committee \\\hline
TB & TeraByte \\\hline
\end{longtable}


\bibliography{lsst,refs_ads,refs,local}


\end{document}
