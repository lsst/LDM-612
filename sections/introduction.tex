\section{Introduction}\label{sec:introduction}

Goal: In narrative language, outline LSST plans and policies for alert distribution for a broad audience.

Audience: LSST science users, community broker developers, funding agencies, LSST Project personnel

Venue:  LDM document (LDM-612), to be posted on the arxiv after CCB acceptance. Potentially a living document

Timeline: draft to SAC by Spring 2018; CCB acceptance by AHM 2018?

{\it Common Questions:} MLG has ported all of her ``commonly asked questions from the community" from the Google Doc into their relevant section of this document.

Note that the SAC had several recommendations for what this broker doc should contain.
\url{https://project.lsst.org/groups/sac/sites/lsst.org.groups.sac/files/2017Aug14_minutes.txt}

Continue to check on the relevant community concerns that are likely to be voiced here:
Community discussion on LSST Alert Broker (categories Science and TVS SC)
\url{https://community.lsst.org/t/lsst-alert-brokers/2286}

Related documents: existing requirements documents (SRD, DMSR, DPDD, etc.) used as input. Future documents might also include:
\begin{itemize}
\item{New Alert Distribution Requirements document (possibly part of L1 Requirements Doc; in formal requirements language)}
\item{New Alert Distribution Test Plan (possibly part of L1 Test Plan)}
\item{New Alert Distribution Design Document (Analogous to LDM-151 for Science Pipelines)}
\end{itemize}

Paragraph of overview for sections?

Conventions used in this document, such as the use of {\tt tt text formatting}, using {\tt Source} for a detection in a single image and {\tt Object} for a given location on the sky, etc. Direct readers to Glossary in Section \ref{sec:glossary}.