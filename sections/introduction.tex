\section{Introduction}\label{sec:introduction}

LSST will discover large numbers of astrophysical objects that move across the sky, change in brightness, or appear as transients.
Because of their intrinsic temporal variability, full scientific exploitation of many classes of these events requires rapid discovery and additional follow-up observations.
Accordingly, LSST's real-time Prompt Processing pipelines will identify such detections using image differencing and report them using world-public \textit{alerts} issued within 60 seconds of the shutter closure after each visit.

These alert packets will contain not only the information about the most recent detection but also a historical lightcurve, cutout images, timeseries features, and other contextual information.
A science user will be able to use the contents of a single LSST alert packet to make a decision that is both rapid and informed about whether the event is relevant to their scientific goals.
LSST expects to produce up to about ten million of these alerts nightly\footnote{The LSST alert stream will contain essentially all DIASources detected at 5$\sigma$ in the difference image, including a currently unknown fraction of artifacts.
LSST will provide a threshold that may be used to filter transients in the alert stream to 90\% completeness and 95\% purity at 6$\sigma$ \citedsp{LSE-30}, so less than 5\% of alerts filtered in this manner will be artifacts.
By varying the cutoff spuriousness value users may adopt other tradeoffs between completeness and purity as well.}.
The resulting \textit{alert stream} will be large---more than 1\,TB nightly---but will contain all classes of astrophysical events, from the youngest supernovae to the most distant RR Lyrae to the slowest-moving Trans-Neptunian Objects to the most unexpected new phenomena.
The task for the scientist is to identify the small subset of events of interest in the larger alert stream.

To do so, astronomers will rely on third party \textit{community brokers}, software systems that receive the full LSST alert stream and provide additional information to refine the selection of events of interest.
Community brokers may crossmatch the LSST stream to multiwavelength catalogs, join LSST alerts with those from other surveys, provide machine-learned classifications of events, and/or offer user filters to winnow the stream.
LSST will itself offer an alert filtering service of more limited capacity that will apply user-supplied filters to the alert stream.
Individual scientists may receive alerts through one or more of these services.

The large volume of the alert stream and the finite bandwidth from the LSST Data Facility necessitate a proposal process to select community brokers that will receive the full stream.
This document outlines the process, criteria, and timeline by which LSST will choose community brokers.
To provide context, we also summarize the major features of the LSST Alert Production systems as well as relevant data rights concerns.

Throughout the document, we provide references to the formal LSST requirements and relevant design documents as an aid to the interested reader.
