\section{Introduction}\label{sec:introduction}

LSST will discover large numbers of astrophysical objects that move, change, or appear.
Because of their intrinsic temporal variability, full scientific exploitation of many classes of these events requires rapid discovery and additional follow-up observations.
Accordingly, LSST's real-time Prompt Processing pipelines will identify such detections using difference imaging and report them using world-public \textit{alerts} issued within 60 seconds of the shutter closure.

These alert packets will contain not only the information about the most recent detection but also a historical lightcurve, cutout images, timeseries features, and other contextual information.
An LSST science user will be able to use the contents of a single alert packet and make an rapid but informed decision about whether the event is relevant to their scientific goals.
LSST expects to produce about ten million of these alerts nightly.
The resulting \textit{alert stream} will be large---more than 1\,TB nightly---but will contain all classes of astrophysical events, from the youngest supernovae to the most distant RR Lyrae to slow-moving Trans-Neptunian Objects.
The task for the scientist is to identify the small subset of events of interest in the larger alert stream.

To do so, astronomers will rely on third party \textit{community brokers}, software systems that receive the full LSST alert stream and add additional information to it.
Community brokers may crossmatch the LSST stream to multiwavelength catalogs, join LSST alerts with those from other surveys, provide machine-learned classifications of events, and/or offer user filters to winnow the stream.
LSST will itself offer a simple filtering service, the ``mini-broker,'' that will apply user-supplied filters to the alert stream.
Individual scientists will receive alerts through one or more of these services.

The large volume of the alert stream and the finite bandwidth from the LSST Data Facility necessitates a proposal process to select community brokers to receive the full stream.
This document outlines the process, criteria, and timeline by which LSST will choose community brokers.
To provide context, we also summarize the major features of the LSST Alert Production systems as well as relevant data rights concerns.

Throughout the document, we provide references to the formal LSST requirements and design documents as an aid to the interested reader.

\textit{
Goal: In narrative language, outline LSST plans and policies for alert distribution for a broad audience.

Audience: LSST science users, community broker developers, funding agencies, LSST Project personnel

Venue:  LDM document (LDM-612), to be posted on the arxiv after CCB acceptance. Potentially a living document

Timeline: draft to SAC by Spring 2018; CCB acceptance by AHM 2018?

{\it Common Questions:} MLG has ported all of her ``commonly asked questions from the community" from the Google Doc into their relevant section of this document.

Note that the SAC had several recommendations for what this broker doc should contain.
\url{https://project.lsst.org/groups/sac/sites/lsst.org.groups.sac/files/2017Aug14_minutes.txt}

The five main requests made by the SAC in August 2017 are:
\begin{itemize}
\item The plans for giving guidelines for and selecting event brokers
  from the scientific community need to reflect the fact there is
  likely to be a continuum of needs: from community brokers designed
  to serve the world, to specialized filters aiming to do specific
  scientific projects.  We recommend that the call for brokers reflect
  this continuum. Guidelines need to be set to describe how many of these specialized
brokers can be accommodated, and what the rules will be.
\item We recommend that the project start a conversation with the
  Transients and Variable Stars science collaboration about use cases
  for brokers, to understand the extent to which the LSST simple
  broker, and a few general community brokers, will or will not
  satisfy the needs of the community. 
\item The SAC remains unclear on whether the outputs of brokers can, or should, be world-public. The SAC remains unclear on exactly what this would include. Will the community brokers have the ability to query the database for additional information (which may be proprietary to the US, Chile, and International Contributors)? If so, do they have the right to share the results with the world? This needs to be clarified.
\item There will also be a desire to run queries on archival data (not just on the real-time alert stream), for statistical studies of various categories of transients.  It remains unclear whether this will be possible.
\item The SAC looks forward to reviewing the document which will describe the requirements and review process for brokers. It should include the proposal process, timescale, and
review process for groups to request access to the raw data stream.
Exactly what will the data stream these brokers will receive include?
What level of support, if any, will the brokers receive from the
Project during construction, commissioning, and operations?  Will the
community or specialized brokers be expected to archive their results
for posterity, and if so, how should/could this be integrated with the
LSST database and science user interface?  Will the answers to these
questions depend on the scale of the broker being proposed?
\end{itemize}

Continue to check on the relevant community concerns that are likely to be voiced here:
Community discussion on LSST Alert Broker (categories Science and TVS SC)
\url{https://community.lsst.org/t/lsst-alert-brokers/2286}

Related documents: existing requirements documents (SRD, DMSR, DPDD, etc.) used as input. Future documents might also include:
\begin{itemize}
\item{New Alert Distribution Requirements document (possibly part of L1 Requirements Doc; in formal requirements language)}
\item{New Alert Distribution Test Plan (possibly part of L1 Test Plan)}
\item{New Alert Distribution Design Document (Analogous to LDM-151 for Science Pipelines)}
\end{itemize}

Paragraph of overview for sections?

Conventions used in this document, such as the use of {\tt tt text formatting}, using {\tt Source} for a detection in a single image and {\tt Object} for a given location on the sky, etc. Direct readers to Glossary in Section \ref{sec:glossary}.
}
