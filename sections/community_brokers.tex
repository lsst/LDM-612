\section{Guidelines for Community Brokers}\label{sec:community_brokers}

\subsection{The Role of Community Brokers}

Community brokers will play a vital role in enabling time-domain science with LSST.
By receiving the LSST alert stream, adding value, and redistributing it to the scientific community, brokers facilitate full exploitation of the scientific value of the real-time LSST alerts.

We envision that community brokers, individually or jointly, may provide a variety of functions.  These may include, but are not limited to:

\begin{itemize}
	\item redistributing alert packets to a defined or world-wide community
	\item filtering alerts for specific science goals or to remove potentially spurious alerts\footnote{trading completeness for purity}
	\item cross-correlating LSST alerts with other static catalogs or alert streams
	\item classifying events scientifically
	\item providing user interfaces to the data
	\item coordinating scientific activity among collaborators
	\item triggering followup observing
	\item for users with appropriate data rights, facilitating followup queries and/or user-generated processing within the LSST Data Access Center
	\item managing annotation \& citation as followup observations are made
	\item collecting classification and other information gathered by the scientific community
\end{itemize}

Some of these functions may be performed by ``Target and Observation Managers'' \citep[TOMs;][]{2018SPIE10707E..11S} that may either be integrated into community brokers directly or connect to annotated and/or filtered streams delivered by a broker.
For simplicity, we will here refer to any system that connects directly to the LSST alert stream as a broker.

\subsection{Requirements for Community Brokers}

Because of the large data volume of the alert stream (several TB per night), finite bandwidth from the LSST datacenter prevents sending a full copy of the alert stream to all interested parties.
Additionally, we anticipate that community brokers will require some level of support from the LSST operations team.
Accordingly, LSST is conducting an open proposal process in order to select brokers with sufficient technical capability to enable valuable and reliable scientific returns.

We believe the scientific community will be best served by a rich broker ecosystem offering diverse capabilities.
Any institution worldwide was allowed to submit a Letter of Intent (\S \ref{sec:LOIs}). 
The LSST SAC decided to invite all institutions who submitted a LOI to submit a full proposal.
Demonstration of appropriate technical and personnel resources to support the proposed goals were the minimum requirement to be considered.

Recognizing that different science goals may benefit from different technical approaches, and that proposing institutions may have a range of strengths and experience, the proposal solicitation does not place any explicit functional requirements on proposed brokers.
Thus a proposed broker is not \textit{required} to consume the whole stream, to redistribute the full stream, to operate for the full ten-year survey, or to make its products world-public, for example.
However, given the small and finite number of brokers to be selected (\S \ref{sec:numbrokers}) and the evaluation criteria (\S \ref{sec:evaluation}), proposals offering to do so will be more likely to be selected.
The selection committee will evaluate the proposals holistically, weighing the unique capabilities proposed in order to maximize the scientific output of LSST.

While we expect the evaluation process to favor proposals that receive the full alert stream, brokers may propose to receive only a filtered subset of events (or a subset of the alert packet contents) from LSST.
We anticipate that this capability would be provided by the same technology underlying the LSST alert filtering service, potentially with less restrictive requirements than are placed on user-provided filters.
Some broker services, particularly those processing only a small subset of alerts, may elect to receive alerts from other community brokers rather than from the LDF directly.
Brokers requesting direct access to the alert stream who agree to forward alerts to downstream services will be favored in the selection process.

\subsection{The Broker Selection Panel}

Broker proposals will be evaluated by a panel designated by the LSST Science Advisory Committee.  Appropriate care will be taken to avoid conflicts of interest.
The panel will include ex officio representatives from the LSST Project Office, LSST Operations, and LSST Data Management to provide policy and technical guidance.

\subsection{Letters of Intent} \label{sec:LOIs}

The broker proposal process will have two stages:
an initial open call for Letters of Intent (LOIs) from all interested parties, and a subsequent full proposal call solicited from invited LOI writers.
This two-stage process ensures those writing full proposals have a reasonable chance at selection and provides an opportunity for early feedback.
An invitation to submit a full proposal may also be valuable to proposers in obtaining relevant funding.
No financial support is available from the LSST Project.
The LOI call was issued in \citeds{LDM-682}.

LOIs were evaluated using the same criteria as the full proposals (\S \ref{sec:resources}--\ref{sec:evaluation}), with recognition that the proposed brokers may have been in the conceptual or early design stage at the time of the LOI.

\subsection{Call for Proposals} \label{sec:CfP}

The call for full proposals for LSST brokers was issued in \citeds{LDM-723}.
The broker selection panel will use these proposals to select the brokers that will receive alert streams directly from the LSST Data Facility.

\subsection{Required Technical Resources} \label{sec:resources}

The panel will evaluate whether the proposed broker system is technically feasible and appropriately supported for the proposed goals.
At minimum, this requires:

\begin{itemize}
	\item Large inbound and outbound network bandwidth (the full alert stream is a few TB/night)
	\item Petabytes of disk capacity
	\item Databases capable of handling billions of sources
	\item Compute resources to handle sophisticated classification and filtering tasks in real time at scale
	\item Appropriate personnel and sufficient effort to develop the required software and to maintain and operate the service
	\item Institutional support to ensure the longevity and stability of the service and software.  By default selected brokers will be expected to plan to operate for the full ten-year survey.
	\item Discussion of funding sources to support broker development and operations.
\end{itemize}

\subsection{Evaluation of Community Broker Proposals} \label{sec:evaluation}

The broker selection panel will evaluate proposals with a primary goal of maximizing the scientific exploitation of LSST---recognizing that many of LSST's scientific returns are likely to be unanticipated.

The panel will evaluate broker proposals that demonstrate appropriate technical resources (\S \ref{sec:resources}) using the following evaluation criteria:

\subsubsection{Scientific Value}

The panel will consider whether a proposed broker has the potential to add scientific value that serves a large community, enables high-profile science, or provides unique capabilities.
In particular, the panel will evaluate the proposed contributions to LSST's four science pillars\footnote{These are: Probing Dark Matter and Dark Energy, Taking an Inventory of the Solar System, Exploring the Transient Optical Sky, and Mapping the Milky Way \citedsp{LPM-17}.}.
The panel will prioritize proposals that require or take advantage of the unique aspects of the LSST alert stream, namely its real-time and world-public nature.

\subsubsection{Availability to the World Community}

Because community brokers are the means by which scientists without LSST data rights may access the public contents of the alert packets, the panel will prefer proposals that make their products and services world-public.
At least one broker that provides world-public access to alerts will be selected.

\subsubsection{Scientific Validity}

The panel will look for evidence that scientific products, such as photometric classifications, produced by a proposed broker are accurate.
Brokers may demonstrate performance in production on precursor surveys or by data challenge.
The proposal call will provide sample alerts generated by LSST pipelines from precursor survey data that may be used for testing and evaluation.

\subsubsection{Integration with the Time-Domain Ecosystem}

For science cases involving real-time followup, the panel will consider the extent to which proposed brokers facilitate integration with followup resources, other surveys and alert streams, other brokers and services, existing communities of observers, and archives.


\subsubsection{Community Adoption}

The panel will consider whether a proposed broker system has already demonstrated scientific value and community adoption on precursor streams.
This evaluation will also assess the
range of scientific, geographic, and institutional communities that have utilized the service as well as the publications it enabled.


\subsubsection{Complementarity}

To maximize scientific returns, the panel will aim to select an ensemble of complementary brokers.
Brokers that are more limited in scope or scientific goals may be selected if they provide capabilities not present in other selected brokers or are serving scientific communities that would otherwise be unable to use LSST alerts.

\subsubsection{Existing Agreements}

The panel will consider the applicability of any existing agreements, if relevant.

\subsection{Determining the Number of Community Brokers} \label{sec:numbrokers}

Early sizing estimates indicated that only four full streams could be provided from the LSST Data Facility due to network bandwidth limitations.
Current capacity projections indicate a larger number may be possible (\S \ref{sec:alert_distribution}), but with some uncertainty.
A minimum of five full streams is required \citedsp{LSE-61}.
The available capacity will depend on:

\begin{itemize}
	\item actual network capacity and usage patterns
	\item details of the alert format
	\item the achieved purity of the alert stream
\end{itemize}

Final capacity determination will not be possible until commissioning data is available.
The broker selection panel may elect to make guaranteed and provisional selections based on these estimates.
A subset of brokers may be selected for integration during the commissioning and early operations phases, with additional brokers joining as the survey reaches steady state.


\subsection{Broker Evolution Throughout Operations}

Broker performance and usage will be reviewed periodically during LSST Operations at an interval to be determined.
Broker selections may be changed or additional new brokers added due to changes in LSST survey or pipeline performance, broker performance or usage, or community science priorities.

\subsection{Resources for Proposers}

To aid in broker development and testing, LSST \oldtext{will make available}\newtext{is providing} pre-packaged alert streams for precursor \oldtext{and/or simulated} datasets generated by the LSST Alert Production pipelines and distributed with the expected formats and protocols.
\newtext{These LSST-processed alerts provide the best match to the content and formats currently envisioned for LSST alerts, although the precursor surveys themselves do not closely match the LSST depth, filters, cadence, etc.
Details on these sample alerts may be found at \url{https://www.lsst.org/scientists/alert-brokers}.

Proposers may also wish to take advantage of the public alert archive from the Zwicky Transient Facility\footnote{\url{https://ztf.uw.edu/alerts/public/}}. 
ZTF alerts are shallower than the sample precursor alerts produced by the LSST pipelines and contain somewhat different information\footnote{ZTF alerts have
30 days of history rather than one year, and no forced photometry or
timeseries features, but do include crossmatches to Gaia and PanSTARRS.}.
However, ZTF alerts are available in much greater quantities and in many
cases have been spectroscopically classified\footnote{see e.g., the Transient
Name Server (\url{https://wiserep-tns.weizmann.ac.il/})}.

Proposers may also or instead utilize simulations or other precursor datasets in their proposals if desired.  
In any case, proposers should be explicit about what data they use for demonstrations, and explain the extent to which, and the limitations to which, they illustrate the scientific and technical capabilities of their system.}

\subsection{Resources for Selected Brokers}

Selected brokers will be expected to sign a Memorandum of Understanding codifying agreement to respect LSST Data Rights policies (where relevant) and outlining expected interfaces, support, and Service-Level Agreements for both parties.
Selected community brokers will receive basic support from the LSST Operations Team to ensure that the alert stream is delivered with the expected form, content, reliability, and timing.
Additionally, status information will be available during operations to diagnose processing and data transport problems as they arise.

\subsection{Timeline}

The calendar-year timeline presented below is the current best estimate based on the LSST project schedule and is subject to revision.
{
	\renewcommand\labelenumi{\textbf{\theenumi}}
\begin{enumerate}
  \setcounter{enumi}{2017}
	\item \begin{itemize} % 2018
		\item \textit{Q4} Plans and Policies for LSST Alert Distribution document issued
	\end{itemize}
	\item \begin{itemize}% 2019
		\item \textit{Q1} Issue call for Letters of Intent (LOI)
		\item \textit{Q2} Letters of intent due
		\item \textit{June} Broker development workshop (invited LOI writers)
		\item \textit{Q4} Revise this document. Invite full proposals for community brokers from selected LOIs
		\end{itemize}
	\item \begin{itemize} %2020
		\item \oldtext{\textit{Q1} Document bulk transport format and interface. Distribute alert stream test software.
		\item June 15 Broker proposals due
		\item \textit{Q3} Provisional broker selections made}
		\item \newtext{\textit{Q3} Document bulk transport format and interface. Distribute alert stream test software.
		\item December 15 Broker proposals due.}
	\end{itemize}
	\item \begin{itemize} %2021
		\item \newtext{\textit{Q2} Provisional broker selections made}
		\end{itemize}
\end{enumerate}
}

Subsequent steps depend on major commissioning milestones and will shift according to the evolving commissioning schedule\footnote{See \url{https://www.lsst.org/about/project-status}.}.
Alerts are provided during commissioning to enable engineering tests of broker interfaces.
Potential data release scenarios are described in \citeds{LSO-011}.

