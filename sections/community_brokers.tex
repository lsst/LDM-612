\section{Guidelines for Community Brokers}\label{sec:community_brokers}

{\it Common Question: Do all brokers need to forward the raw alert stream publicly? This is also asked in Section \ref{sec:components}. Relatedly, people ask more generally about the specific minimum or required functionality that brokers must provide.}

{\it Common Question: How many brokers will be awarded and when will the process be defined?} Obviously going to be answered here.

\subsection{Why are there requirements on community brokers?}

Networking resources - Each full alert stream requires 1 gbs bandwidth

Staffing resources - It is anticipated that community brokers will require some level of support from the LSST operations team

Science integrity ? We would like reliability for the community

World-public ? We would like to prioritize resource investment into brokers that will serve a broad community

\subsection{How many Community Brokers are expected?}

Initial sizing estimates supported only four full streams due to network bandwidth limitations.

Current capacity projections indicate a larger number may be possible, but with some uncertainty:

\begin{itemize}
	\item Depends on actual network capacity and usage patterns

	\item Depends on details of the alert format

	\item Depends on the achieved purity of the alert stream
\end{itemize}

Don?t think we can make a final decision on the number of brokers until we have at least processed data from ComCam.


\subsection{What is the planned proposal process?}

\subsection{Who is eligible to propose for a community broker?}

\subsection{Who is eligible to propose for a community broker?}

\subsection{How will proposed brokers be evaluated?}

Demonstration of availability of appropriate technical and personnel resources
\begin{itemize}
	\item Large inbound and outbound network bandwidth (full alert stream is a few TB/night)
	\item Petabytes of disk capacity
	\item Databases handling of billions of sources
	\item Compute resources to handle sophisticated classification and filtering tasks in real time at scale
	\item Personnel to develop and maintain the service
\end{itemize}


Potential to add scientific that serves a large community, enables high-profile science, and/or provides unique capabilities

Willingness to make products world public (preferred)

Integration with follow-up resources and/or the broader time-domain ecosystem

Demonstrated performance and/or community adoption on precursor streams

Demonstration of scientific validity on precursor data or by data challenge

Applicability of international agreements, if relevant


\subsection{What is the anticipated timeline?}

March 2018: present draft policy document to SAC for feedback

August 2018: obtain Project approval for policy document

Mid 2018: Set up test alert stream with sample precursor data

2019: VO interactions to develop format

2020: Call for proposals for community brokers

	Sample commissioning alerts with ComCam \textit{with substantial latency}

2021: Finalize number of full streams

	Sample commissioning alerts with LSSTCam \textit{with substantial latency}

	Selection of LSST brokers for early operations

2022: Begin operations

It is reasonable to expect periodic review of broker performance and usage during LSST Operations.


\subsection{Must community brokers consume the whole stream?}

\subsection{Must community brokers serve the whole stream?}

\subsection{Must community brokers redistribute alerts to everyone?}

