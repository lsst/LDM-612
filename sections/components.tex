\section{Components and Capabilities of the LSST Alert Distribution System}\label{sec:components}

In this section we provide a high-level overview of the LSST Alert Production process to provide background for science users of LSST \texttt{Alerts}.

\subsection{Data Acquisition, Transfer, and Processing}

LSST will survey the sky repeatedly using standard or alternative standard visits ($2\times15$\,second or $1\times30$\,second exposures, respectively).  
Immediately after the camera shutter closes, the images will be transferred via fiber networks up to the LSST Data Facility (LDF) at the National Center for Supercomputing Applications (NCSA).
All subsequent data processing leading to \texttt{Alerts} occurs in the LDF.

\subsection{The LSST Data Facility and Data Access Center}

The LSST Data Facility (LDF) is hosted at 
Brokers might want to incorporate additional LSST information outside of what is available in the {\tt Alert} packet.
This section describes the basic data products and their availability timelines, and how they can be accessed.
All access to proprietary LSST data will be via the LSST Data Access Center (DAC), administered by NCSA.
This is the case for both Project and Science Community members.
Access will be through authorized user accounts that are subject to authentication protocols, and all accounts will be subject to caps on the amount of computational resources available for query, processing, and storage.
{\bf MLG: I think there's not yet a proper reference for the DAC?} The following data products and services will be available through the DAC.
For a full description of the LSST data products, see \citeds{LSE-163}.

\subsubsection{Alert Production Pipeline}\label{sec:AP}

The LSST Data Management (DM) Alert Production (AP) pipeline processes data during the night and generates {\tt Alerts}.
It is described in \citeds{LSE-163}, with detailed descriptions of its algorithmic components provided in \citeds{LDM-151}.
The basic modules include single-frame image processing (e.g., instrument signature removal, photometric and astrometric calibration), difference imaging analysis (DIA; the subtraction of a template image), and source detection, association, and measurement \citedsp{LSE-163}.
The AP pipeline will process all standard and alternative standard visits, whether obtained as part of the main wide-fast-deep survey or a Special Program \citeds{DMTN-065} and generate {\tt Alerts} within the required {\tt OTT1 = 1} minute after the shutter closes (DMS-REQ-0004, \citeds{LSE-61})\footnote{
The AP pipeline may also be able to process non-standard visits with longer or shorter exposure times provided that, e.g., the visit image can be successfully PSF-matched and differenced with a template image.
\texttt{Alert} generation in crowded fields may produce more than the maximum of 10,000 \texttt{Alerts} per visit required to be supported by DM; \texttt{Alert} generation in this circumstance is still under study.
}.

% MLG draft
All sources in a difference image that have a signal-to-noise ratio ${\rm SNR}>5$ in positive or negative flux are considered ``detected" and incorporated into the {\tt DIASource} catalog, and will cause an {\tt Alert} to be issued\footnote{
The may be exceptions to this rule, such as the following two examples.
(1) Sources with ${\rm SNR} > 5$ that have a {\it ``high probability of being instrumental non-astrophysical artifacts"} \citedsp{LSE-163}, potentially as determined by 
a to-be-deeveloped spuriousness or real/bogus classifier \citedsp{LDM-151}.
(2) Sources with ${\rm SNR} < 5$ that meet other to-be-determined criteria, such as the likelihood of being a potentially hazardous asteroid.
}.
Additionally some ${\rm SNR} < 5$ sources will be kept for diagnostic purposes, but will not lead to {\tt Alerts}.
Source association and measurement occurs prior to {\tt Alert} generation: every {\tt DIASource} will have one unique match to a {\tt DIAObject} (stationary object) or known Solar System Object {\tt SSObject} (moving object; see Section \ref{sssec:AGP_MOPS}).
The stationary source association algorithms will be probabilistic and incorporate motion models for parallax and proper motion \citedsp{LDM-151}.
If no association is possible a new {\tt DIAObject} is created.
Source measurements including centroid, fluxes, shapes, and other characterization parameters (see e.g., Table 1 of \citeds{LSE-163}) are made, and the time-evolving parameters of {\tt DIAObject} such as the parallax, mean flux, and periodic/non-periodic light curve features, are updated to include the new associated {\tt DIASource}.
At this point, the {\tt DIASource} is used to create an {\tt Alert} packet, which is described in Section \ref{ssec:packets}.

%%%MLG commented out the following statements on Tue Apr 3.
% Deblending for difference image sources will only be run if necessary (\citeds{LSE-163}, page 11). 
%One exception is sources with $SNR>5$ but also a {\it ``high probability of being instrumental non-astrophysical artifacts"} \cite{LSE-163}. The application of real/bogus (spuriousness) algorithms might only be done if the fraction of false positives is greater than 50\% (\citeds{LDM-151}; OSS-REQ-0354) {\bf (but how would that be known without first applying real-bogus?)}. Rejected sources might not be persisted (i.e., not included in the {\tt DIASource} catalog and no {\tt Alert} is triggered). \citeds{LDM-151} states that the default technique will be {\it ``based on a trained random forest classifier ... conditioned on the image quality and airmass of the observation''}, but otherwise the spuriousness algorithm is very sparsely defined in \citeds{LDM-151}: {\it ``Some per-source measure of likelihood the detection is junk. May use machine learning on other measurements or pixels. May be augmented by spuriousness measures that aren't purely per-source"} (page 118-11). Another exception is a small fraction of high-priority sources with $SNR<5$ that meet some set of criteria, such as $SNR>3$ and near one of Earth's gravitational keyholes (i.e., a potentially hazardous asteroid), would be persisted and lead to {\tt Alerts} (Section 3.2.1, \citeds{LSE-163}). Additionally, some $SNR<5$ {\tt DIASources} will be kept for diagnostic purposes but not lead to {\tt Alerts}.
% It is important to note that, especially in crowded fields, there might be instances of {\tt DIASource} association reassignment as, e.g., motion models improve over time. {\tt DIASources} will also be matched by location to the 3 nearest stars and 3 nearest galaxies in the {\tt Objects} catalog from the last data release with a probability of association provided for each (\citeds{LDM-151}).
%{\it Common Question: Will Alerts be released after 60 or 120 seconds?} The requirement governing the latency of reporting on optical transients -- the time between shutter close and {\tt Alert} distribution -- is {\tt OTT1 = 1} minute (DMS-REQ-0004, \citeds{LSE-61}). {\bf Other relevant quantities such as GB processed per minute, number of {\tt DIASources} per visit, number of {\tt DIAObjects} survey-total, etc.?}

\subsubsection{Moving Objects Processing System}\label{sssec:AGP_MOPS}


As part of Alert Generation, the Known Solar System Object Assocation pipeline will calculate the predicted locations of known solar system objects (the LSST {\tt SSObjects} catalog) in each newly acquired visit image. 
{\tt Alerts} will be issued for all ${\rm SNR} > 5$ \texttt{DIASource} detections of \texttt{SSObjects}.

After the completion of a night of observing,
the LSST Moving Objects Processing System (MOPS; \citeds{LDM-156}) will attempt to identify new \texttt{SSObjects} using the updated Prompt Products Database (\S \ref{sec:PPDB}).
MOPS algorithms form tracklets from single-apparition \texttt{DIASources} taken during one night, generate tracks between nights, and fit orbits to identify and characterize moving objects.
Newly identified moving objects are added to the {\tt SSObject} catalog.
MOPS will interface with the Minor Planets Center (MPC) and will ingest all previously known or externally identified moving objects into the {\tt SSObjects} catalog, and use MPC astrometry in the orbital parameter fits.


\subsection{Alert Packets}\label{ssec:packets}

The contents of an {\tt Alert} packet are fully described in Section 3.5.1 of \citeds{LSE-163}. We reproduce the list of included data here:
\renewcommand{\labelenumi}{\Roman{enumi}.}
\begin{enumerate}
\item an ID uniquely identifying the {\tt Alert}
\item the prompt products database ID
\item the {\tt DIASource} record that triggered the {\tt Alert}
\item the entire associated {\tt DIAObject} or {\it SSObject} record from the prompt products database
\item the previous 12 months of associated {\tt DIASource} records from the prompt products database
\item matching {\tt Object} IDs from the latest Data Release, if they exist, and 12 months of their {\tt DIASource} records
\item postage stamps of the difference image and template at the {\tt DIASource} location
\end{enumerate}

The full list of parameters that are measured and included in the {\tt DIASource} and {\tt DIAObject} records are provided in Tables 1 and 2 of \citeds{LSE-163}, respectively.
%The average size of an {\tt Alert} packet will be {\bf XXX?} kilobytes and {\bf XXX?} \% of this is the postage stamp.
The only trigger for an {\tt Alert} is the detection of a source in a difference image with ${\rm SNR} > 5$. 
Thus, objects that become saturated or return to the same brightness as in the template image will not generate an {\tt Alert}, for example.

\subsection{Alert Distribution}

\texttt{Alert} packets will be queued for distribution to community brokers (\S \ref{sec:community_brokers}) and the LSST mini-broker (\S \ref{sec:mini-broker}). 
LSST is prototyping  a bulk transport system built on the open-source distributed queue system Apache Kafka, with Apache Avro used as a binary serialization format.
Initial tests indicate that this system performs effectively at the required scale \citeds{DMTN-028}.

Due to finite bandwidth out of the LSST Data Center, only selected community brokers will receive the full \texttt{Alert} stream.
Science users may access LSST alerts through a community broker, or through the mini-broker if they have data access rights (\S \ref{sec:data_rights}).



\subsubsection{Forced Photometry}\label{sssec:AGP_force}

Measurements of \textit{DIAObjects} that are below the ${\rm SNR} > 5$ threshold are availble through ``forced photometry'': flux is measured at a previously-known position.
Forced photometry is performed on difference images in two ways; neither contributes to \texttt{Alerts}, but are available to users through the Science Platform.
First, during the real-time DIA, forced photometry is performed on every difference image at the locations of all previously-known {\tt DIAObjects} detected in the past 12 months. 
The resulting measurements are stored in the Prompt Products Database (\S \ref{sec:prompt_products}).
Second, at the end of the night, ``precovery'' forced photometry will be performed for all \textit{new} \texttt{DIAObjects} on every difference image from the past 30 days.
These results are also stored in the Prompt Products database and will be available within 24 hours.
 
Additionally, a service shall be provided to users to obtain full-survey precovery for a limited number of {\tt DIAObjects}, also within 24 hours (DMS-REQ-0341, \citeds{LSE-61}).
This service will be made available through the Science Platform.

\subsubsection{Prompt Processing Products}

Products of the prompt processing pipeline will be available in the DAC within 24 hours, and potentially much sooner.

%Products related to {\tt Alert} generation (Section \ref{sssec:AGP}), such as processed single-visit images, difference images, and updated versions of the {\tt DIASource} and {\tt DIAObject} catalogs, become available with the same latency as {\tt Alert} delivery: 1 minute.
%Forced photometry and MOPS run during the day and updated versions of the relevant catalogs are available before sunset (Sections \ref{sssec:AGP_force}, \ref{sssec:AGP_MOPS}).
%It is important to note that the prompt catalogs are limited to a $\sim12$ month window (i.e., include data since the most recent annual release), and that e.g., variability characterization parameters, are only calculated for that time window.

\subsubsection{Data Release Products}

Annual releases of all the LSST data will include processed and stacked images, catalogs of {\tt Sources}, {\tt ForcedSources}, and {\tt Objects} from measurements on the stacked and individual images, as well as calibration information.
It will also include a  reprocessing of all images with the latest DIA codes and full-survey versions of {\tt DIASource} and {\tt DIAObject} catalogs -- e.g., the variability characterization parameters are calculated from the survey-to-date.

\subsubsection{Alert Database}

All {\tt Alerts} will be stored in their full original form in a database that can be queried.
These queries can be combined with the aforementioned data products of the prompt and data release pipelines.
The {\tt Alerts} database is updated in real time as {\tt Alerts} are issued.

\subsubsection{The LSST mini-broker}\label{sssec:mini-broker}

\subsubsection{The LSST Science Platform}

The LSST Science Platform (LSP) is described in full in \citeds{LSE-319} and \citeds{LDM-554}.
It will serve as a portal to the data, toolkit for analysis, and processing power.
The most relevant for brokers is probably web application programming interface (API).


