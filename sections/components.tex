\section{Components and Capabilities of the LSST Alert Distribution System}\label{sec:components}

\subsection{Relevant aspects of the fiber networks, NCSA, and Alert Generation pipelines.}

\subsection{Alert Packets}

Contents and triggering criteria.

{\it Common Question: What do Alerts contain? The full past light curve, non-detections, the host photo-z, etc. Obviously what we're going to answer here anyway.}

{\it Common Question: What is the size of the Alert packet and what fraction of that is image stamp?}

{\it Common Question: Will Alerts be released after 60 or 120 seconds?}

{\it Common Question: Requests for descriptions of all cases that trigger Alerts in addition to a SNR>5 DIASource. For example: telescope slews, a source reaching saturation, the forced photometry done for all DIAObjects with detections in past 12 months.}

{\it Common Question: What will be the format of source association to e.g., DRP Objects (id or full record).}



\subsection{Alert Transport}

{\it Common Question: How will an interrupted stream recuperate from a slow-down? Will there be, for example, no alerts issued once they're older than 300s? (Probably the public will want all of the Alerts, even with comprised latency).}

{\it Common Question: How behind will the Alert Stream get during visits to crowded fields?} MLG notes that the answer might be ``not at all" if elastic computational resources are deployed at NCSA (convos with K.T.).


\subsection{Community Brokers}

{\it Common Question: Do all brokers need to forward the raw alert stream publicly? This is also asked in Section \ref{sec:community_brokers}.}


\subsection{The LSST DAC}

\subsubsection{Processed Images and Catalogs}

\subsubsection{L1 DB}

\subsubsection{Alert DB}

\subsubsection{The LSST mini-broker}

\subsubsection{The LSST Science Platform}
