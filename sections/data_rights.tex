\section{Data Rights to Alert Stream Components}\label{sec:data_rights}

\textbf{LPM-216 is not yet change-controlled and so the following should be regarded as provisional until it is.  Broken references are to LPM-216.}
Data rights and access policies are formally described in \citeds{LPM-216}. 
Here we summarize the aspects most relevant to the alert stream.
The interested reader is advised to consult \citeds{LPM-216} for details.

LSST {\tt Alert} packets (\S \ref{sec:packets}) are a world-public data product.
We use ``public'' here in the sense of DAPOL-020:  \texttt{Alert} packets can be freely shared with anyone, by anyone, anywhere, worldwide.
However, LSST is not committed to serving alert packets directly to individuals without data rights;
thus, the term ``public'' means ``shareable'' and should not be misinterpreted as ``freely available.''

Instead, the LSST {\tt Alert} Stream will be delivered to a to-be-determined set of community brokers (Section \ref{sec:community_brokers}).
An institution is not required to hold data rights in order to host a broker.
Brokers may share (or not share) the contents of {\tt Alert} packets with whomever they choose, 
although the broker selection process (\S \ref{sec:community_brokers}) will prioritize selection of brokers that will provide world-public access.
Brokers will not be prohibited from storing {\tt Alerts} and making them available at later times;
the {\tt Alert} packets themselves are world-public indefinitely, and thus aggregations of public products are also permitted.

Essentially all other LSST data products and services are proprietary, including raw and processed images, the resulting catalogs and databases, and access to the LSST DAC, Science Platform, and mini-broker.

Brokers hosted by institutions with LSST data rights and data access (DAPOL-040) may access proprietary LSST data products, such as through API interfaces exposed by the LSST Science Platform (\S \ref{sec:LSP}).
The broker would then be responsible for ensuring access to the proprietary data products is restricted to data rights holders (DAPOL-020).
However, brokers may create and make public \textit{derived} data products that use proprietary data.
DAPOL-620 defines derived data products as data that cannot be used to recreate any proprietary \textbf{ECB: draft says ``original''} LSST data products.
Derived data products may be shared without restriction.

For  example,  a community broker could use DAC APIs to obtain the proprietary full-history forced-photometry light curve for a \texttt{DIAObject} identified by a public {\tt Alert}, 
and then use this information to construct a classification probability that represents the likelihood the object is a tidal disruption event.
The classification probability can be shared publicly without restriction, but the light curve itself must remain proprietary.

%{\it A common question about broker DAC privileges: Will there be a way for brokers to make products available through the DAC, so that authorization (in the case of brokers using proprietary data products) can be handled by LSST's protocols?} Case scenario: a broker runs an associated user-generated pipeline/script that does forced photometry on the prompt single-visit process images when a previously variable {\tt DIAObject} is undetected at ${\rm SNR}<5$.
%Furthermore, brokers that are accessing the DAC through a web API during the night might use a large amount of the DAC's computational resources.
%{\bf MLG: Has this been sized?
%Is it possible?
%How would decisions to throttle usage be made, if it became necessary to do so?}


Full LSST Users (DAPOL-080) will have access to the mini-broker in the LSST DAC for filtering the alert stream.
Since the alerts themselves are world public, a Full LSST User who uses the mini-broker to process the alert stream can export filtered alert packets (or any subset of their contents) and share them freely.
Full LSST Users may use other DAC services to query and process proprietary data products. 
For example, a user might identify a \texttt{DIAObject} of interest from a mini-broker filter, query the Prompt Products Database to retrieve the precovery forced photometry, and then run user-generated processing on the Processed Visit Images.
These proprietary products must not be distributed to individuals without data rights.
However, as for community brokers, if a user generates derived data products (such as a classification probability derived from a fit to the forced photometry lightcurve), these may be freely shared.

As an example, we consider a Full LSST user who is searching for supernovae using the mini-broker.
The LSST user wants to share a target with a collaborator who does not have data rights so that the collaborator may obtain a spectrum.
The LSST user may send their collaborator the sky position and time of the first LSST detection (public from the \texttt{Alert} packet), the alert packet cutout images (public from the \texttt{Alert} packet), and their assessment that the target is a likely supernova Ia (derived data product).
They may not share a larger cutout from the Processed Visit Image to use as a finder chart nor the forced photometry lightcurve they used to derive the supernova classification (proprietary data products).
